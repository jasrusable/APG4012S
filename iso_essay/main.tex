\documentclass[11pt,a4paper]{article}

\usepackage[backend=bibtex]{biblatex}
\addbibresource{sources.bib}

\title{ISO certifications applicable to Geomatics businesses}
\date{2015}
\author{Jason Russell}

\begin{document}

\maketitle

\newpage

The use of standards aids in the creation of products and services that are safe, reliable and of good quality.  Standards also facilitate interoperability and cohesion between products by ensuring consistency.

\vspace{5mm}

The International Organization for Standardization (ISO) is an international standard-setting body composed of representatives from various national standards organizations. The organization promotes worldwide proprietary, industrial and commercial standards. It was founded in 1947 and is headquartered in Geneva. ISO is an independent, non-governmental organization.

\vspace{5mm}

Standards help businesses increase productivity while minimizing errors and waste. The ISO offers a number of various standards which are designed to be applied to businesses. Some of the popular ISO businessess standards include ISO 9000 - Quality Management, ISO 31000 - Risk Management and ISO 26000 - Social Responsability.

\vspace{5mm}

Geomatics encompases a number of industries including surveying, cartogrophy, remote sensing and geographic information services. Because of the general nature of these industries, it can be advantagous for companies to conform to formal ISO standards.


\newpage
\printbibliography

\end{document}
