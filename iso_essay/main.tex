\documentclass[11pt,a4paper]{article}

\usepackage[backend=bibtex]{biblatex}
\addbibresource{sources.bib}

\title{ISO certifications applicable to Geomatics businesses}
\date{2015}
\author{Jason Russell}

\begin{document}

\maketitle

\newpage

The use of standards aids in the creation of products and services that are safe, reliable and of good quality.  Standards also facilitate interoperability and cohesion between products by ensuring consistency. This essay is concerned with standards applicable to geomatics businesses, specific attention will be paid to the accessibility, cost and relevance of these standards.

\vspace{5mm}

The International Organization for Standardization (ISO) is an international standard-setting body composed of representatives from various national standards organizations. The organization promotes worldwide proprietary, industrial and commercial standards. It was founded in 1947 and is headquartered in Geneva. ISO is an independent, non-governmental organization.

\vspace{5mm}

Standards help businesses increase productivity while minimizing errors and waste. The ISO offers a number of various standards which are designed to be applied to businesses. Some of the popular ISO businesses standards pertaining to geomatics businesses include ISO 9000 - Quality Management, ISO 31000 - Risk Management and ISO 26000 - Social Responsibility. These standards will be individually discussed in detail below as they are of particular relevance to geomatics firms.

\vspace{5mm}

<<<<<<< HEAD
The ISO 9000 is the range of standards which are geared towards quality management systems. Their purpose is to facilitate and enable consistency of services and products, this ensures that customer requirements are met and are consistent. The ISO 9000 family is administered by certification and accreditation bodies. Some of the benefits of ISO 9000 include improved productivity and efficiency, cost reductions, competitive advantage and increased marketing and sales opportunities, improved consistency of product and service performance, higher customer satisfaction levels, and improved customer perception \cite{_get_????}. In order to a business to make use of ISO 9000, a business must register the following: define what they do, document what they do, do what they say they do, and do this effectively. Depending on the complexity of an organisation, it can take between four to eighteen months to become ISO 9000 certified.
=======
Geomatics encompases a number of industries including surveying, cartogrophy, remote sensing and geographic information services. Because of the general nature of these industries, it can be advantagous for companies to conform to formal ISO standards.
>>>>>>> a3e8569303be0e5b60274707f022195f6c775d52

\vspace{5mm}

The ISO 31000 standards range is geared towards risk management and risk prevention and mitigation. ISO 31000 provides principles, frameworks and processes for managing risk, there are no concrete requirements due to the nature of the standard. The standard can be implemented into any organization, regardless of its size, complexity, activity or sector. The aim of using ISO 31000 is to increase the likelihood of achieving objectives, improve the identification of opportunities and threats and effectively allocate and use resources for risk treatment. \cite{_iso_????}

\vspace{5mm}

The ISO 26000 range provides guidance on how businesses and organizations can operate in a socially responsible way. This includes ethical and transparency guidelines which enable the health and welfare of society. The aim of the standard is to clarify social responsibility, help businesses and organizations turn principles into actions. The standard is aimed at any organization, regardless of their activity, size or sector. The standard was launched in 2010 and was contributed to by representatives from various agencies including governments, NGOs, consumer groups and labour organizations from around the world. 

\vspace{5mm}

Geomatics encompasses a number of industries including surveying, cartography, remote sensing and geographic information services. The surveying industry is made up of many sub industries, including cadastral, mining, hydro-graphic and engineering surveying. The products of these industries is often required to be of high quality and precision and any deficiencies can result in catastrophic results both financially, as well as morally. Thus it can be seen as critical for certain standards to be met. Because of the general nature of these industries, it can be advantageous for companies to conform to formal standards of some kind. Many geomatics businesses are linked in that they work together to provide a service or product to a client. An example of this would be a survey company acquiring raw data from measurements and sub-contracting a GIS firm to further analyze and interpret the data. It would therefore be beneficial to both companies that each comply with a set of standards in order to aid interoperability and efficiency. In this scenario, if both companies were to comply with ISO standards, the GIS firm can be assured that the data was acquired in a socially responsible manner, and the survey firm can be assured that the end product form the GIS firm will meet quality standards. 

\vspace{5mm}

In order to acquire ISO certification, a number of steps need to be undertaken by an organization. Often, an internal or external consultant who has experience in the particular field of the certification should be employed in order to further implement and advise on certification matters. This consultant should be familiar with the particular ISO standard and should be part or a reputable and professional and accredited body. ISO audits are conducted in various stages over time. Stage one includes completing the particular standards manual which aims to introduce the organization to the requirements of the standard. The consultant will then check that your written manual will meet the requirements of the standard and validate that what has been stated in the manual is true. The consultant will also then highlight any areas of deficiency or areas where potential improvements can be made within the manual. Once the manual has been approved, stage two certification assessment can begin. This stage involves inspection of businesses functions of compliance to the particular standard. This stage may be conducted by an independent third party accredited certification body, registered with ISO.

\vspace{5mm}

In order to make effective use of any standard, it is important for an organization to have clearly defined goals and objectives, and clearly have identified what benefits will have been gained by certifying. All organizational staff must be aware of what is expected of them, their responsibilities should be clear and documented in order to meet certification requirements. Audits and training should be conducted regularly in order to ensure full compliance with a standard. There are a number of firms which offer this as a service.

\vspace{5mm}

The cost of ISO certification can vary a lot depending on the consultant, employee or certification body used. The first step in becoming certified is to get a system developed and implemented. This can be done in one of two ways. The first option would be to send internal staff for training which can cost between R15,000 and R40,000 depending on the courses the staff are to attend. The staff also need to gain experience once the training is completed, and so this could also be a timely and costly process. The second option would be to contract the services of a  professional consultant to provide the service of certification. This could cost anywhere in the range of R30,000 to R70,000. Some of the pros of this option is that the consultant would have immediate knowledge and experience to start getting certification ready right away. The consultant will, on most occasions, work with and train an internal employee so that no extra salary costs would be applicable. Factors generally influencing certification costs include the company size, number of staff, complexity of the business, operating sector, working hours and hourly rates etc. Most consultants charge a monthly rate which will end as soon as certification is achieved. Aside from the cost of training or consultation, there is a cost in maintaining a certification. This cost can range from R45,000 to R100,000 for a three year certification. In the fourth year, one will have to apply for recertification and then again pay the same amount as was paid in the first year. So this means that there is a constant cost involved in maintaining certification. Factors which influence this cost include again the company size, number of branches, number of staff, audit days etc. In total, the cost of certification implementation can range from R70,000 to R150,000 \cite{_iso_????-1}. When it comes down to selecting a certification body, a number of things should be considered. It is important to evaluate several certification bodies in order to determine the most cost effective implementation. The certification body should use the relevant CASCO standard. One should also ensure that the certification body is accredited. Accreditation is not mandatory, however accreditation does provide independent confirmation of competence. \cite{_iso_????-2}

\vspace{5mm}

Many South African geomatics businesses are highly regulated by a number of professional bodies who function as an ad hoc standards body. These bodies are often themselves regulated by laws. For example, all practicing professional land surveyors are required by law to be registered with the PLATO organization. The PLATO organization is bound by various acts and regulations. The main function of the PLATO organization is to govern over conduct and professionalism of its members. This in turn ensures an acceptable level of service for all clients and minimizes deficiencies in the work of PLATO members. In the case of land surveyors with the PLATO body, as well as other geomatics professions with their respective bodies, the need for standardization through ISO is not as great as it may be for businesses in fields without similar professional bodies. As a result of this, and due to the high relative cost of associated ISO certification fees, many small South African geomatics businesses do not have ISO certification as it is deemed unnecessary. Their certification comes in the form of conforming to the requirements of the respective professional bodies which impose their own restrictions and standards on their members. Due to the nature of how many small 
geomatics businesses operate and the nature of the services they provide, many of the benefits of ISO standards and certifications are unneeded and irrelevant. For example, there is no need for further standardization of the products of land surveyors such as ER diagrams and noting sheets. This is mainly due the the esoteric nature of the field as well the ubiquity of existing systems and standards. \cite{_south_????}

\vspace{5mm}

In contrast to what was mentioned above, Furgo is an international engineering design and infrastructure company and is one of the largest surveying companies in the world. Fugro offers services which include geotechnical services, survey services, subsea services and seabed geosolutions. Furgro is certified with ISO standards 9000, 134000 27000. Fugro claims that meeting these requirements establishes a systematic approach to quality management and ensures that all clients needs are clearly understood agreed and fulfilled \cite{_iso_????-1}. It is very important for Fugro to meet the requirements of international standards such as those of the ISO as the client base is composed of a wide range of clients from all over the world, and with varying requirements and standards.

\vspace{5mm}

In conclusion, while acquiring ISO certification is accessible in South Africa, for many local geomatics firms, it is not practical. ISO certification is not practical in that it may be seen as superfluous due the existing professional bodies whose job it is to regulate their respective professions. Additionally, the relatively high cost of acquiring ISO certification may not be justifiable, especially for smaller geomatics firms such as privately practicing land surveyors. One solution to this particular cost problem might be for professional bodies to provide and document guidelines and their own standards, which may conform to those of ISO. This would ensure high levels of consistency and competency throughout respective fields, as well as familiarity to similar ISO standards.

\newpage
\printbibliography

\end{document}
